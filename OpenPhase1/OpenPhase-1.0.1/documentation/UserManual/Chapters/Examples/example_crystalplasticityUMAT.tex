\section{Crystal plasticity UMAT}
\label{sec:example_cpumat}
The CrystalPlasticityUMAT benchmark is designed to compare the CrystalPlasticity module of OpenPhase with results of CP codes such as DAMASK \citeopref{Roters2012}. It evaluates the stresses and hardening variables at a single continuum point for a given strain increment. This approach is analogue to the evaluation of an integration point under a given deformation gradient (or strain) in finite element frameworks such as Abaqus (where the user-defined material model is described in a so called UMAT).

\subsection{Modules and Parameters}
The benchmark uses the three mandatory modules \nameref{sec:module_phasefield}, \nameref{sec:module_boundaryconditions} and \nameref{sec:module_settings}. In order to perform the mechanical calculations it is necessary to load \nameref{sec:module_orientations}, \nameref{sec:module_elasticproperties} as well one of the two homogenization modules \nameref{sec:module_elasticityreuss} or \nameref{sec:module_elasticitykhachaturyan}, respectively. For the plasticity part PlasticPropertiesCPphenom and PlasticityModelCPphenom have to be loaded.\\
The benchmark sets up a one point phase field with a single phase/grain present. By defining an applied strain $\bar{\varepsilonB}$ the full stress tensor
\begin{equation}
	\sigmaB = \mathcal{C} : \left(\bar{\varepsilonB} - \varepsilonB_p\right)
\end{equation}
the plastic strain increment
\begin{equation}
	\varepsilon_p = f(\sigmaB, \dot{\gamma}^{(\alpha = 1..12)}, \tau_c^{(\alpha = 1..12)})
\end{equation}
(see crystal plasticity model for more details) as well as the hardening
\begin{equation}
	\tau_c^{(\alpha = 1..12)} = f(\sigmaB, \dot{\gamma}^{(\alpha = 1..12)})
\end{equation}
for each glide system $\alpha$ is calculated. Since the plasticity model is of viscous type, parameter $\Delta t$ is a crucial input parameter. The domain size, adjusted with parameter $dx$ can be chosen arbitrarily, since the length-scale will not enter the mechanical solution (at least for a local model). The calculation will return an error, if the plasticity could not converge for a given $\bar{\varepsilonB}$. The grain orientation can be adjusted by \cmethod{Orientations::GrainEulerAngles[0].set()}. For completeness, a set of important input parameters is given in table~\ref{tab:tablePICPUMAT}. 
\begin{table}[ht]
\centering
\begin{tabular}{lll}
\toprule
Key & Value & Comment \\
\midrule
\$nSteps & 1 & Single step\\
\$Nx & 1 & \multirow{3}{8cm}{Note: Only a single point is evaluated.}\\
\$Ny & 1 & \\
\$Nz & 1 & \\
\$Nphses & 1 & Single phase\\
\$dt & 1 & \\
\$dx & 1 & \\
\bottomrule
\end{tabular}
\caption{ProjectInput.opi for CrystalPlasticityUMAT}
\label{tab:tablePICPUMAT}
\end{table}
\subsection{Results}
All material parameters are given in \si{\milli\meter, \second, \kilogram} units.
\begin{table}[ht]
\begin{center}
\subfloat[Elasticity input for CrystalPlasticityUMAT benchmark]{
\begin{tabular}{lcc}
\toprule
Key & Value 1  & Value 2 \\
\midrule
\$Phase 0\\
\$C11 & 280,000 &  0\\
\$C22 & 280,000 &  0\\
\$C33 & 280,000 &  0\\
\$C12 & 120,000 &  0\\
\$C13 & 120,000 &  0\\
\$C23 & 120,000 &  0\\
\$C44 & 80,000 &   0\\
\$C55 & 80,000 &  0\\
\$C66 & 80,000 &  0\\
\bottomrule
\end{tabular}}
\qquad
\subfloat[Plasticity input for CrystalPlasticityUMAT benchmark]{
\begin{tabular}{lc}
\toprule
Key & Value\\
\midrule
\$explicitSolver    & No\\
~\\
\$plasticityflag\_0  & Yes\\
\$nonlocalflag\_0    & No\\
\$gamma0\_0          & $0.001$\\
\$tauc\_0            & $10.0$\\
\$taus\_0            & $117.0$\\
\$h0\_0              & $100.0$\\
\$flowPow\_0         & $10.0$\\
\$hardPow\_0         & $2.25$\\
\bottomrule
\end{tabular}}
\end{center}
\caption{Set of material parameters used for benchmark calculation.}
\end{table}
