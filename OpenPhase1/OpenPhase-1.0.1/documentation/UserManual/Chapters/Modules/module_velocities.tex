\section{Velocities}
\label{sec:module_velocities}
% Define box and box title style
\tikzstyle{mybox} = [draw=black, fill=RUBgrau!20, very thick, rectangle, rounded corners, inner sep=10pt, inner ysep=20pt]
\tikzstyle{fancytitle} =[fill=RUBblau!100, text=white, rectangle, rounded corners,draw=black]

\begin{tikzpicture}
\node[mybox] (box)
{
  \begin{minipage}{1\textwidth}
  \begin{description}
  \item[What it does] Stores velocity field of phases and calculates average.
  \item[Requires] \nameref{sec:module_settings}, \nameref{sec:module_phasefield}, \nameref{sec:module_boundaryconditions}
  \item[Header file] Velocities.h
  \item[Input file] None
  \item[Examples] /examples/LDPlasticity, ...
  \end{description}
  \end{minipage}
};
\node[fancytitle, right=12pt] at (box.north west) {Module in brief...};
\end{tikzpicture}
\paragraph{Background and usage} The main purpose of the Velocities module is the storage of a velocity field, here called $\mathrm{V}(i,j,k)$

The temperature field can be read and write to disk using \textsf{Write(int tStep)} and \textsf{Read(tStep)}. VTK output can be generated via \textsf{WriteAverageVTK(int tStep)} for the phase averaged quantity.

% ----------------------------------------------------------------
\CallGraphSettings

\begin{figure}
\centering
\begin{tikzpicture}[framed, node distance = 2cm, auto]
    \node [block] (constructor) {constructor};
    \node [block, below of=constructor] (init) {Initialize()};
 
    % Draw edges
    \path [line] (constructor) -- (init);
\end{tikzpicture}
\caption{In and output for module \nameref{sec:module_velocities}}
\end{figure}
