\section{Plasticity modules}
\label{sec:module_plasticity}
\begin{tikzpicture}
\node[mybox] (box)
{
\begin{minipage}{1\textwidth}
\begin{description}
\item[What it does]
\item[Requires] \nameref{sec:module_settings}, \nameref{sec:module_phasefield}, \nameref{sec:module_boundaryconditions}
\item[Input file] PlasticPropertiesJ2.opi,PlasticPropertiesCPphenom.opi, etc.
\item[Examples]
\end{description}
\end{minipage}
};
\node[fancytitle, right=12pt] at (box.north west) {Module in brief...};
\end{tikzpicture}
\subsection*{Structure}
The plasticity functionality of OpenPhase is splitted to classes storing and managing the corresponding storages (\cmethod{PlasticProperties, PlasticPropertiesJ2, ...}) and classes to calculate the plastic strain (\cmethod{PlasticityModel, PlasticityModelJ2, ...}). In a typical elasto-plastic calculation the properties and the model classes have to be invoked and are passed to the SpectralElastoPlasticSolver. 
\subsection*{Homogenization}
Plastic strains and plastic properties (hardening parameters, dislocation densities, etc.) are defined for each individual phase-field $\gamma$. Using
\begin{equation}
	\varepsilonB^{\mathrm{pl}}_{\mathrm{eff}} =\sum_\gamma^{N_\gamma} \varepsilonB^{\mathrm{pl}}_\gamma \phi_\gamma
\end{equation}
the complete homogenization (Reuss $\sigmaB = \sigmaB_\alpha = \sigmaB_\beta$) becomes
\begin{equation}
  \varepsilonB_{\mathrm{eff}} = \left(\sum_\gamma^{N_\gamma}\phi_\gamma\mathcal{\CB}_{\gamma}^{-1}\right)\sigmaB +\underbrace{\sum_\gamma^{N_\gamma} \varepsilonB^*_\gamma}_{\varepsilonB^*_\mathrm{eff}} + \underbrace{\sum_\gamma^{N_\gamma} \varepsilonB^{\mathrm{pl}}_\gamma}_{\varepsilonB^\mathrm{pl}_{\mathrm{eff}}}.
\end{equation}
The effective values are handed over to the spectral elastic solver.
\subsection*{J2 plasticity model}
The J2 model represent the most simple plasticity model implemented in OpenPhase. It is an isotropic plasticity model with linear and kinematic hardening. 
Yielding occurs, if 
\begin{equation*}
f(\sigmaB, \qB) = \|\etaB\|-\sqrt{\frac{2}{3}}(\textcolor{red}{\sigma_y} + \textcolor{red}{\theta} \textcolor{red}{\bar{H}}\alpha)
\end{equation*}
where 
\begin{equation*}
\etaB~\hat{=} \mathrm{dev}(\sigmaB) - \betaB.
\end{equation*}
The stress deviator is defined as
\begin{equation*}
\mathrm{dev}(\sigmaB) = \sigmaB - \frac{1}{3}\mathrm{trace}(\sigmaB)
\end{equation*}
Furthermore, $\sigma_y$ is the yield strength obtained by unidirectional tensile tests, $\theta$ and $\bar{H}$ 
  \begin{eqnarray*}
     \etaB~\hat{=}& \mathrm{dev}(\sigmaB) - \betaB\\
	\dot{\varepsilonB}^{p} =& \gamma \frac{\etaB}{\|\etaB\|}\\
	\dot{\alpha} = &\gamma\sqrt{\frac{2}{3}}\\
	\dot{\betaB} = &\gamma \frac{2}{3} (1-\theta)\textcolor{red}{\bar{H}}(\alpha)\frac{\etaB}{\|\etaB\|}
  \end{eqnarray*}
\subsection*{Hill plasticity model}
In opposite to the J2 model, the Hill model allows an anisotropic yield surface. Currently no hardening model is implemented for this model.
Starting with
\begin{equation*}
\dot{\varepsilon}^p = \dot{\lambda}\frac{\partial f}{\partial \sigmaB}
\end{equation*}
a yield surface is defined as
\begin{equation*}
f = \alpha_{12}(\sigmaB_{11} - \sigmaB_{22})^2 + \alpha_{23}(\sigmaB_{22}-\sigmaB_{33})^2 + \alpha_{31} (\sigmaB_{33} - \sigma_{11})^2 + 6\alpha_{44}\sigmaB_{12}^2+6\alpha_{55}\sigmaB_{23}^2+6\alpha_{66}\sigmaB_{31}^2
-\bar{\sigma}^2.
\end{equation*}
$\alpha_{12}$, $\alpha_{23}$, $\alpha_{31}$, $\alpha_{44}$, $\alpha_{55}$,$\alpha_{66}$ are the anisotropic parameters with respect to a reference yield strenght $\bar{\sigma}$. Briefly, one could also write
\begin{equation*}
f = \frac{3}{2}\sigmaB^T  \PB \sigmaB - \bar{\sigma}^2
\end{equation*}
where
\begin{equation*}
\PB = 
\begin{pmatrix}
\frac{1}{3}(\alpha_{12}+\alpha_{31}) & -\frac{1}{3}\alpha_{12} & -\frac{1}{3}\alpha_{31} & 0 & 0 & 0 \\
-\frac{1}{3}\alpha_{12} & \frac{1}{3}(\alpha_{23}+\alpha_{12}) & -\frac{1}{3}\alpha_{23} & 0 & 0 & 0 \\
-\frac{1}{3}\alpha_{31} & -\frac{1}{3}\alpha_{23} & \frac{1}{3}(\alpha_{31}+\alpha_{23}) & 0 & 0 & 0 \\
0 & 0 & 0 & 2\alpha_{44} & 0 & 0 \\
0 & 0 & 0 & 0 & 2\alpha_{55} & 0 \\
0 & 0 & 0 & 0 & 0 & 2\alpha_{66} \\
\end{pmatrix}
\end{equation*}
The Hill model will fall back to the J2 model for $\alpha_{12} = \alpha_{23} = \alpha_{31} = \alpha_{44} = \alpha_{55} = \alpha_{66} = 1$.

More details can be found in \citeopref{deBorst1990}.

\subsection*{Crystal plasticity model}
The FCC crystal plasticity model starts from 
\begin{gather}
	\DB^{\mathrm{pl}}_\gamma =\sum_{s=1}^{N_s}{\dot{\gamma}^s_\gamma  \frac{1}{2}\left(\mB^s_\gamma\otimes\nB^{s}_\gamma + \nB^s_\gamma\otimes\mB^{s}_\gamma\right)} = \sum_{s=1}^{N_s}{\dot{\gamma}^s_\gamma \PB^s}\\
	\WB^{\mathrm{pl}}_\gamma = \sum_{s=1}{\dot{\gamma}^s_\gamma \frac{1}{2}\left(\mB^s_\gamma\otimes\nB^{s}_\gamma - \nB^s_\gamma\otimes\mB^{s}_\gamma\right)} = \sum_{s=1}^{N_s}{\dot{\gamma}^s_\gamma \MB^s}\\
	\shortintertext{where}
	\begin{tabular}{>{$}r<{$}@{\ :\ }l}
		 \dot{\gamma}^s_\gamma & shear rate on glide system $s$\\ 
		 \DB^{\mathrm{pl}}_\gamma & plastic strain rate in phase $\gamma$\\ 
		 \WB^{\mathrm{pl}}_\gamma & plastic spin in phase $\gamma$\\ 
   		 \mB^{s}_\gamma & vector of the slip direction\\
   		 \nB^{s}_\gamma & glide system's normal direction
   \end{tabular}\nonumber
\end{gather}
The phenomenological crystal plasticity model uses the following model instead
\begin{gather}
	\dot{\gamma}^s_\gamma = \gamma_0 \left(\frac{|\sigmaB_\gamma:\MB^s_\gamma|}{\tau_{c,\gamma}^s}\right)^{m_\gamma} \mathrm{sgn}\left(\sigmaB_\gamma:\MB^s_\gamma\right),
	\shortintertext{where}
	\begin{tabular}{>{$}r<{$}@{\ :\ }l}
   		 \gamma_0 &  referential shear rate\\
   		 \sigmaB_\gamma:\MB^s_\gamma & resolved shear stress on glide system $s$\\
   		 \tau_{c,\gamma}^s & critical resolved shear stress on glide system $s$\\
   		 m_\gamma & hardening exponent 
   \end{tabular}\nonumber
\end{gather}
The evolution of the critical resolved shear stress is assumed to be
\begin{equation}
  \dot{\tau}_{c,\alpha} = q_{\alpha\beta} h_0 \left(1-\frac{\tau_\beta}{\tau_s}\right)^a |\dot{\gamma}_\beta|
\end{equation}
Herein, $h_0$ is a referential hardening parameters, $\tau_s$ is the saturation stress and $a$ the hardening exponent. Finally, $q_{\alpha\beta}$ is a $12\times12$ matrix $1.0$ for self-hardening terms and $1.4$ for co-planar hardening. In FCC materials, the glide systems are defined by <111> glide plane normals and [100] glide directions, see Table \ref{tab:glidesystemsFCC}\\

\renewcommand{\kbldelim}{.}
\renewcommand{\kbrdelim}{.}
\begin{table}
\hspace{1.5cm}
\kbordermatrix{\mbox{GS}&
n1& n2 & n3 & d1 & d2 & d3\\
1& -1.0 & 1.0 & 1.0 & 0.0 & 1.0 &-1.0\\
2& -1.0 & 1.0 & 1.0 & 1.0 & 0.0 & 1.0\\
3& -1.0 & 1.0 & 1.0 & 1.0 & 1.0 & 0.0\\
4&  1.0 & 1.0 & 1.0 & 0.0 & 1.0 &-1.0\\
5&  1.0 & 1.0 & 1.0 & 1.0 & 0.0 &-1.0\\
6&  1.0 & 1.0 & 1.0 & 1.0 &-1.0 & 0.0\\
7&  1.0 & 1.0 &-1.0 & 0.0 & 1.0 & 1.0\\
8&  1.0 & 1.0 &-1.0 & 1.0 & 0.0 & 1.0\\
9&  1.0 & 1.0 &-1.0 & 1.0 &-1.0 & 0.0\\
10& 1.0 &-1.0 & 1.0 & 0.0 & 1.0 & 1.0\\
11& 1.0 &-1.0 & 1.0 & 1.0 & 0.0 &-1.0\\
12& 1.0 &-1.0 & 1.0 & 1.0 & 1.0 & 0.0
}
\caption{Glide systems in FCC material}
\label{tab:glidesystemsFCC}
\end{table}

For an dislocation based approach, the slip rate is defined  as
\begin{gather}
	\dot{\gamma}^s_\gamma = \rho^s_\gamma b_\gamma v_{0,\gamma} \left(\frac{|\sigmaB_\gamma:\MB^s_\gamma|}{\tau_{c,\gamma}^s}\right)^{m_\gamma} \mathrm{sgn}\left(\sigmaB_\gamma:\MB^s_\gamma\right),
	\shortintertext{where}
	\begin{tabular}{>{$}r<{$}@{\ :\ }l}
   		 v_{0,\gamma} &  referential dislocation velocity\\
   		 b_\gamma & length of the burgers vector\\
   		 \sigmaB_\gamma:\MB^s_\gamma & resolved shear stress on glide system $s$\\
   		 \tau_{c,\gamma}^s & critical resolved shear stress on glide system $s$\\
   		 m_\gamma & hardening exponent 
   \end{tabular}\nonumber
\end{gather}
The evolution of statistically stored dislocations (SSD) due to micro-mechanical processes is governed by the Kocks-Mecking evolution law
\begin{equation}
	\dot{\rho}_{\gamma}^s = \left(c_1 \sqrt{\rho^s_\gamma} - c_2 \rho^s_\gamma\right)|\dot{\gamma}_\gamma^s|
\end{equation}

\subsection*{Integration}
The velocity gradient $\LB$ is returned from Fourier space as
\begin{equation}
  \widetilde{\LB}(\xiB) = i \xiB \otimes\widetilde{\uB}(\xiB).
\end{equation}
and can be decomposed additively to strain rate and spin
\begin{equation}
 \LB = \DB + \WB = \DB^{\mathrm{el}} + \WB^{\mathrm{el}} + \DB^{\mathrm{pl}} + \WB^{\mathrm{pl}}
\end{equation}
Following the integration scheme as introduced by [Peirce, Asaro, Needleman, Acta metall. 1983], one gets
\begin{equation}
  \sigmaB_{n+1} = \sigmaB_{n} + \left(\sigmaB_{n+1}^\nabla + \WB_{n+1}\sigmaB_{n} - \sigmaB_{n}\WB_{n+1}\right) \delta t.
\end{equation}
where the Jaumann rate of Kirchhoff stress reads
\begin{equation}
  \sigmaB^\nabla_{n+1} = \mathcal{\CB}_{n}\DB_{n+1}-\sum_{s=1}^{N_s}{\RB}_n\dot{\gamma}_s - \sigmaB_n \mathrm{tr}(\DB_{n+1})
\end{equation}
and
\begin{equation}
\RB_s = \mathcal{\CB} \PB_s + \WB \sigmaB - \sigmaB \WB_s
\end{equation}
All local tensors (stiffness, eigenstrains, diffusion parameter, ...) are rotated from the initial configuration to the current orientation by $\OB_{n+1}$. Only the elastic spin $\WB^{\mathrm{el}} = \WB - \WB^{\mathrm{pl}}$ rotates the crystal, hence 
\begin{equation}
 \OB_{n+1} = \exp\left(\WB^{\mathrm{el}}\delta t\right)\OB_{n}
\end{equation}
where
\begin{equation}
\exp\left(\WB^{\mathrm{el}}\delta t\right) = \IB + \frac{\sin \left(\|\delta t\WB^{\mathrm{el}}\|\right)}{\|\WB^{\mathrm{el}}\|}+\frac{1-\cos \left(\|\delta t\WB^{\mathrm{el}}\|\right)}{\|\WB^{\mathrm{el}}\|^2}\WB^{\mathrm{el}}\WB^{\mathrm{el}}
\end{equation}