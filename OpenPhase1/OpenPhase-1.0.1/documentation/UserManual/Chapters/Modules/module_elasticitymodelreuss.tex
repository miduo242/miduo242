\section{ElasticityReuss}
\label{sec:module_elasticityreuss}

% Define box and box title style
\tikzstyle{mybox} = [draw=black, fill=RUBgrau!20, very thick, rectangle, rounded corners, inner sep=10pt, inner ysep=20pt]
\tikzstyle{fancytitle} =[fill=RUBblau!100, text=white, rectangle, rounded corners,draw=black]

\begin{tikzpicture}
\node[mybox] (box)
{
\begin{minipage}{1\textwidth}
\begin{description}
\item[What it does] Calculates homogenized elastic properties in the interface; calculates elastic driving force.
\item[Requires] \nameref{sec:module_settings}, \nameref{sec:module_elasticproperties}
\item[Header file] Mechanics/ElasticityModels/ElasticityReuss.h
\item[Input file] None
\item[Examples] /examples/MultiComp-Elastic
\end{description}
\end{minipage}
};
\node[fancytitle, right=12pt] at (box.north west) {Module in brief...};
\end{tikzpicture}
The \cvar{EffectiveElasticConstants} are calculated by calling the method \cmethod{SetEffectiveElasticConstants(EP)} as
\begin{equation}
  \CB = \left(\sum_\alpha \CB^{-1}_\alpha  \phi_\alpha \right)^{-1}
\end{equation}
The \cvar{EffectiveEigenStrains} are calculated by calling the method \cmethod{SetEffectiveEigenStrains(EP)}
\begin{equation}
  \varepsilon^* = \sum_\alpha \phi_\alpha \varepsilonB^*_\alpha
\end{equation}
The driving force $\Delta G^{el}_{\alpha\beta}$ becomes
\begin{align}
  \Delta G^{el}_{\alpha\beta} &= \frac{1}{2} \varepsilonB^{el} \CB \left(\CB_\alpha^{-1}-\CB_\beta^{-1}\right)\CB \varepsilonB^{el} + \varepsilon^{el} \CB \left(\varepsilonB^*_\alpha-\varepsilonB^*_\beta\right)\\
  &= \frac{1}{2} \sigmaB \left(\CB_\alpha-\CB_\beta\right)\sigmaB - \sigmaB \left(\varepsilonB^*_\alpha-\varepsilonB^*_\beta\right)
  \label{eq:module_elasticityreuss_drivingforce}
\end{align}

% ----------------------------------------------------
\CallGraphSettings

\begin{figure}
\centering
\begin{tikzpicture}[framed, node distance = 2cm, auto]
    \node [block] (constructor) {\cmethod{Constructor}};
        \node [block, below of=constructor] (constants) {\cmethod{SetEffectiveElasticConstants(ElasticProperties)}};
    \node [block, below of=constants] (eigenstrains) {\cmethod{SetEffectiveEigenStrains(ElasticProperties)}};
    
    % Draw edges
    \path [line] (constructor) -- (constants);
    \path [line] (constants) -- (eigenstrains);
\end{tikzpicture}
\caption{Call graph for module \nameref{sec:module_elasticityreuss}}
\end{figure}
