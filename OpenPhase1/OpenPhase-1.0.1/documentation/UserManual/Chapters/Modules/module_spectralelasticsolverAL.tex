\section{SpectralElasticSolverAL}
\label{sec:module_spectralelasticsolverAL}

% Define box and box title style
\tikzstyle{mybox} = [draw=black, fill=RUBgrau!20, very thick, rectangle, rounded corners, inner sep=10pt, inner ysep=20pt]
\tikzstyle{fancytitle} =[fill=RUBblau!100, text=white, rectangle, rounded corners,draw=black]

\begin{tikzpicture}
\node[mybox] (box)
{
\begin{minipage}{1\textwidth}
\begin{description}
\item[What it does] Calculates the mechanical equilibrium $\nabla \cdot \sigma = 0$.
\item[Requires] \nameref{sec:module_settings}, \nameref{sec:module_phasefield}, \nameref{sec:module_boundaryconditions}, \nameref{sec:module_elasticproperties}
\item[Input file] ElasticProperties.opi
\item[Examples] /examples/EshelbyTest
\end{description}
\end{minipage}
};
\node[fancytitle, right=12pt] at (box.north west) {Module in brief...};
\end{tikzpicture}
\paragraph{Background} The SpectralElasticSolverAL module represents an alternative to the \nameref{sec:module_spectralelasticsolver} and the \nameref{sec:module_spectralelasticsolverBS}. It implements the scheme as described by \citeopref{Michel2000} using an enhanced mathematical formulation (augmented lagrangian or AL), strong phase contrast can be treated with improved convergence behavior, see \citeopref{Moulinec2014} for a comparison. However, the AL scheme uses, like the BS solver, a strain related and hence can not be used in combination with the \nameref{sec:module_largedeformations} module.

\begin{equation}
  \eB^i = \varepsilonB + (\CB^0 + \CB)^{-1}\left(\lambda^{-1}-\CB:\varepsilonB\right)
\end{equation}

The Green operator for anisotropic materials reads in Fourier space
\begin{equation}
  \hat{\Gamma}^0_{ijkl} = -k_j k_l \left(C^0_{ijkl} k_l k_j \right)^{-1} 
\end{equation}

The iteration procedere is stopped, when
\begin{equation}
 \max \left(\frac{\| \varepsilonB^i - \eB^i\|}{\|\EB\|}, \frac{\|\lambda^i-\frac{\partial w}{\partial \varepsilon}(\varepsilon^i)\|}{\|\langle \frac{\partial w}{\partial \varepsilon}(\EB)\rangle\|} \right) \leq \eta
\label{eq:convAL}
\end{equation}

\paragraph{Usage} The solver is called via \cmethod{Solve(ElasticProperties,
            double equilibrium, int MAXIterations)}, where \textsf{equilibrium} is the convergence criterium following (\ref{eq:convAL}). From our experience values of $\eta = 10^{-5}$ or $\eta = 10^{-6}$ yield good accuracy. \cvar{MAXIterations} sets the maximum allowed number of iterations, which inherently depends severly on the stiffness contrast of the used materials. Before the solver is called, the load is internally finalized with \cmethod{SetLoad(ElasticProperties)}.Check wether  \cmethod{SetGrainsProperties(...)}, \cmethod{SetEffectiveElasticConstants()} and \textsf{ SetEffectiveEigenstrains(...)} of \cmethod{ElasticProperties} and \cmethod{ElasticityModel} have been called.
                     
            
% ----------------------------------------------------
\CallGraphSettings

\begin{figure}
\centering
\begin{tikzpicture}[framed, node distance = 2cm, auto]
    \node [block] (constructor) {\cmethod{Constructor}};
    \node [block, below of=constructor] (init) {\cmethod{Initialize(...)}};
    \node [block, below of=init] (load) {\cmethod{SetLoad(ElasticProperties)}};
    \node [block, below of=load] (solve) {\cmethod{Solve(...)}};
 
    % Draw edges
    \path [line] (constructor) -- (init);
    \path [line] (init) -- (load);
    \path [line] (load) -- (solve);
\end{tikzpicture}
\caption{Call graph for module \nameref{sec:module_spectralelasticsolverAL}}
\end{figure}