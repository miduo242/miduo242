\section{BoundaryConditions}
\label{sec:module_boundaryconditions}

% Define box and box title style
\tikzstyle{mybox} = [draw=black, fill=RUBgrau!20, very thick, rectangle, rounded corners, inner sep=10pt, inner ysep=20pt]
\tikzstyle{fancytitle} =[fill=RUBblau!100, text=white, rectangle, rounded corners,draw=black]

\begin{tikzpicture}
\node[mybox] (box)
{
\begin{minipage}{1\textwidth}
\begin{description}
\item[What it does] Sets boundary conditions for scalar and vectorial calculation variables.
\item[Requires] \nameref{sec:module_settings}
\item[Header file] BoundaryConditions.h
\item[Input file] BoundaryConditions.opi
\item[Examples] /examples/HeatEquationSolver
\end{description}
\end{minipage}
};
\node[fancytitle, right=12pt] at (box.north west) {Module in brief...};
\end{tikzpicture}
Boundary Conditions can be set individually for every side in the boundary condition input file (typically BoundaryConditions.opi) which is read by \cmethod{ReadInput()}. The keywords \cmethod{\$BC0X}, \cmethod{\$BCNX}, \cmethod{\$BC0Y}, \cmethod{\$BCNY}, \cmethod{\$BC0Z} and \cmethod{\$BCNZ} define the six individual sides of the calculation domain, see Listing \ref{lst:BCInputListing}

\lstinputlisting[frame=single,float=h,captionpos=b, caption= Example of BoundaryConditions.opi input file for BoundaryConditions, label=lst:BCInputListing]{Chapters/Modules/listing_bcinput.tex}

The available types for a scalar or vectorial variable $\xi$
\begin{description}
 \item[Periodic] $\xi_1 = \xi_N$
 \item[Free] $\xi_1, \xi_N$
 \item[NoFlux] $(\nabla \xi_1)\nB = (\nabla \xi_N)\nB = 0$
 \item[Fixed] $\xi_1, \xi_N$
\end{description}
Internally, a boundary condition object containing the boundary definitions is passed to the indivual modules, such as \nameref{sec:module_temperature} or \nameref{sec:module_phasefield}.

% ------------------------------------------------------------------------------
\CallGraphSettings

\begin{figure}
\centering
\begin{tikzpicture}[framed, node distance = 2cm, auto]
    \node [block] (constructor) {\cmethod{Constructor}};
    \node [block, below of=constructor] (init) {\cmethod{Initialize(...)}};
        \node [block, below of=init] (readinput) {\cmethod{ReadInput(BoundaryConditions.opi)}};
    
    % Draw edges
    \path [line] (constructor) -- (init);
    \path [line] (init) -- (readinput);
\end{tikzpicture}
\caption{Call graph for module \nameref{sec:module_boundaryconditions}}
\end{figure}
