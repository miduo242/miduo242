\section{SpectralElastoPlasticSolver}
\label{sec:module_spectralelastoplasticsolver}

% Define box and box title style
\tikzstyle{mybox} = [draw=black, fill=RUBgrau!20, very thick, rectangle, rounded corners, inner sep=10pt, inner ysep=20pt]
\tikzstyle{fancytitle} =[fill=RUBblau!100, text=white, rectangle, rounded corners,draw=black]

\begin{tikzpicture}
\node[mybox] (box)
{
\begin{minipage}{1\textwidth}
\begin{description}
\item[What it does] Calculates the mechanical equilibrium $\nabla \cdot \sigma = 0$.
\item[Requires] \nameref{sec:module_settings}, \nameref{sec:module_phasefield}, \nameref{sec:module_boundaryconditions}, \nameref{sec:module_elasticproperties}, \nameref{sec:module_plasticproperties}, \nameref{sec:module_plasticitymodule} 
\item[Input file] ElasticProperties.opi
\item[Examples] /examples/EshelbyTest
\end{description}
\end{minipage}
};
\node[fancytitle, right=12pt] at (box.north west) {Module in brief...};
\end{tikzpicture}
\paragraph{Background} This module extends \nameref{sec:module_spectralelasticsolver} for consideration of plastic deformations.


% ------------------------------------------------------------------------------
\CallGraphSettings

\begin{figure}
\centering
\begin{tikzpicture}[framed, node distance = 2cm, auto]
    \node [block] (constructor) {\cmethod{Constructor}};
    \node [block, below of=constructor] (init) {\cmethod{Initialize(...)}};
    
    % Draw edges
    \path [line] (constructor) -- (init);
\end{tikzpicture}
\caption{Call graph for module \nameref{sec:module_spectralelastoplasticsolver}}
\end{figure}
