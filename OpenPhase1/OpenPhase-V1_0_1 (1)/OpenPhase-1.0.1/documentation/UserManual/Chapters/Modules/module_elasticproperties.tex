\section{ElasticProperties}
\label{sec:module_elasticproperties}

% Define box and box title style
\tikzstyle{mybox} = [draw=black, fill=RUBgrau!20, very thick, rectangle, rounded corners, inner sep=10pt, inner ysep=20pt]
\tikzstyle{fancytitle} =[fill=RUBblau!100, text=white, rectangle, rounded corners,draw=black]

\begin{tikzpicture}
\node[mybox] (box)
{
\begin{minipage}{1\textwidth}
\begin{description}
\item[What it does] Defines and manages storages for mechanical calculation
\item[Requires] \nameref{sec:module_settings}
\item[Header file] Mechanics/Storages/ElasticProperties.h
\item[Input file] ElasticityInput.opi
\item[Examples] /benchmarks/EshelbyTest /examples/LD-J2Plasticity
\end{description}
\end{minipage}
};
\node[fancytitle, right=12pt] at (box.north west) {Module in brief...};
\end{tikzpicture}

\paragraph{Input parameters and input file}
Several parameters can be defined in the input file for the ElasticProperties module and read via \cmethod{ReadInput()}. An example is given in Listing \ref{lst:ElasticInputListing}. 
\begin{description}
 \item[0 - Pressure relaxation mode] This boundary condition is only supported in \nameref{sec:module_spectralelasticsolver} and \nameref{sec:module_spectralelastoplasticsolver}.
 \item[1 - Applied strain mode] This boundary condition is supported by all solvers.
 \item[2 - Applied stress mode] This boundary condition is supported by all solvers.
 \item[3 - Mixed stress strain mode] This boundary condition is supported by \nameref{sec:module_spectralelasticsolverBS} and \nameref{sec:module_spectralelasticsolverAL}.
\end{description}
\lstinputlisting[frame=single,float=p,captionpos=b, caption= Example of ElasticProperties.opi input file for ElasticProperties, label=lst:ElasticInputListing]{Chapters/Modules/listing_elasticinput.tex}

% -------------------------------------------------------------------------

\CallGraphSettings

\begin{figure}
\centering
\begin{tikzpicture}[framed, node distance = 2cm, auto]
    \node [block] (constructor) {\cmethod{Constructor}};
    \node [block, below of=constructor] (init) {\cmethod{Initialize(...)}};
        \node [block, below of=init] (readinput) {\cmethod{ReadInput(ElasticProperties.opi)}};
    \node [block, below of=readinput] (setgrainsproperties) {\cmethod{SetGrainsProperties(PhaseField)}};
    
    % Draw edges
    \path [line] (constructor) -- (init);
    \path [line] (init) -- (readinput);
    \path [line] (readinput) -- (setgrainsproperties);
\end{tikzpicture}
\caption{Call graph for module \nameref{sec:module_elasticproperties}}
\end{figure}
