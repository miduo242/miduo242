\section{Settings}
\label{sec:module_settings}
% Define box and box title style
\tikzstyle{mybox} = [draw=black, fill=RUBgrau!20, very thick, rectangle, rounded corners, inner sep=10pt, inner ysep=20pt]
\tikzstyle{fancytitle} =[fill=RUBblau!100, text=white, rectangle, rounded corners,draw=black]

\begin{tikzpicture}
\node[mybox] (box)
{
\begin{minipage}{1\textwidth}
\begin{description}
    \item[What it does] Reads and stores the most general calculation informations, mostly given in the input file. 
    \item[Header File] Settings.h
    \item[Requires] Nothing
    \item[Input file] ProjectInput.opi\footnotemark
    \item[Examples] /examples/HeatEquationSolver, ...
\end{description}
\end{minipage}
};
\node[fancytitle, right=12pt] at (box.north west) {Module in brief...};
\end{tikzpicture}
\footnotetext{All following input file names in the 'Module in brief' boxes can be altered by the user.}
\paragraph{Input parameters and input file} The settings module reads in the calculation parameters from an input file, typically ProjectInput/ProjectInput.opi. The following list provides an explanation of the required input parameters. Mandatory entries are underlined.
\begin{description}
 \item[\underline{\$Nx}, \underline{\$Ny}, \underline{\$Nz}] Sets the number of grid points in the x, y and z direction, respectively. The minimum dimension size is one, the maximum is typically limited by the amount of memory. Can be accessed via \cvar{Settings.Nx}.
 \item[\$nSteps] Number of time steps. Accessible via \cvar{Settings.nSteps}. The use of this parameter can be convenient in a typical simulation setup such as 
\begin{lstlisting}[frame=single]
for(int tStep = 0; tStep < Settings.nSteps; tStep++)
{
	// increment operations
}
\end{lstlisting}
 \item[\$FTime, \$FDisk] Sets parameters useful for the amount of data output to disk and screen. They can be accessed via the variables \cvar{Settings.tFileWrite} and \cvar{Settings.tScreenWrite}. Usefully, the 
\begin{lstlisting}[frame=single]
if (!(tStep%Settings.tFileWrite))
{
    PhaseField.WriteVTK(tStep);
}
\end{lstlisting}

\item[\underline{\$dx}] Sets the distance between two calculation points of the regular grid. Note, that this value is the same for all spacial directions. Can be accessed via \cvar{Settings.dx}.
\item[\underline{\$IWidth}] Definition of the interface with in grid points which is stored as \cvar{Settings.eta}. Note that the small strain elasto-plastic framework of OpenPhase also works in the sharp interface limit, i.e. $\cvar{Settings.eta} = 0$.
\item[\underline{\$dt}] Sets the initial time step. Can be accessed via \cvar{Settings.dt}. Note that some modules might use an internal time stepping scheme.
\item[\underline{\$nOMP}] Sets the maximum number of used threads for parallelization with OpenMPI \citeopref{open_mpi}. The domain is then splitted along the x-axis, which should be considered when non-cubic computation grids are created. In addition, some initializations are specifically optimized for this domain decomposition.
\item[\underline{\$nPhses}] Sets the number of thermodynamic phases (but not the number of grains/phase-fields) and can be accessed via \cvar{Settings.nPhases}. Typically, material parameters have to be given for \cvar{Settings.nPhases} in further input files such as ElasticProperties.opi.
\end{description}


% -------------------------------------------------------------------------

\CallGraphSettings

\begin{figure}
\centering
\begin{tikzpicture}[framed, node distance = 2cm, auto]
    \node [block] (constructor) {\cmethod{Constructor}};
    \node [block, below of=constructor] (init) {\cmethod{Initialize(...)}};
        \node [block, below of=init] (readinput) {\cmethod{ReadInput(Settings.opi)}};
    
    % Draw edges
    \path [line] (constructor) -- (init);
    \path [line] (init) -- (readinput);
\end{tikzpicture}
\caption{Call graph for module \nameref{sec:module_settings}}
\end{figure}
